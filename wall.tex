\documentclass{sciposter}


\usepackage{epsfig}
\usepackage{amsmath}
\usepackage{amssymb}
\usepackage{multicol}
%\usepackage{fancybullets}

\newtheorem{Def}{Definition}

%\definecolor{BoxCol}{rgb}{0.9,0.9,0.9}
% uncomment for grey background to \section boxes 
% for use with default option boxedsections

%\definecolor{BoxCol}{rgb}{0.9,0.9,1}
% uncomment for light blue background to \section boxes 
% for use with default option boxedsections

%\definecolor{SectionCol}{rgb}{0,0,0.5}
% uncomment for dark blue \section text 






\title{Reminder}

% Note: only give author names, not institute
\author{Lei Ma}
 
% insert correct institute name
\institute{Department of Physics and Astronomy,\\
           University of New Mexico\\}

\email{leima@unm.edu}  % shows author email address below institute

%\date is unused by the current \maketitle


% The following commands can be used to alter the default logo settings
%\leftlogo[0.9]{logoWenI}{  % defines logo to left of title (with scale factor)
%\rightlogo[0.52]{RuGlogo}  % same but on right

% NOTE: This will require presence of files logoWenI.eps and RuGlogo.eps, 
% or other supported format in the current directory  
%%%%%%%%%%%%%%%%%%%%%%%%%%%%%%%%%%%%%%%%%%%%%%%%%%%%%%%%%%%%%%%%%%%%%%%%%%%%%%%%
%%% Begin of Document



\begin{document}
%define conference poster is presented at (appears as footer)

\conference{{\bf 2015-11-10}}

%\LEFTSIDEfootlogo  
% Uncomment to put footer logo on left side, and 
% conference name on right side of footer

% Some examples of caption control (remove % to check result)

%\renewcommand{\algorithmname}{Algoritme} % for Dutch

%\renewcommand{\mastercapstartstyle}[1]{\textit{\textbf{#1}}}
%\renewcommand{\algcapstartstyle}[1]{\textsc{\textbf{#1}}}
%\renewcommand{\algcapbodystyle}{\bfseries}
%\renewcommand{\thealgorithm}{\Roman{algorithm}}

%\maketitle

%%% Begin of Multicols-Enviroment
\begin{multicols}{3}

%%% Physics

\section{Physics}

\subsection{Dimensions}



\subsection{Limits of The Results}


\subsection{Good to Remember}

\begin{itemize}
\item Energy and temperature: \begin{equation}
\frac{1}{40}\mathrm{eV} = 300\mathrm{K} k_B .
\end{equation}
\item Natural units: energy is related to length by
\begin{equation}
1\mathrm{fm}\times 197\mathrm{MeV}=\hbar c = 1.
\end{equation}
\item For light, energy 1eV corresponds to wavelength $1.24\mathrm{\mu m}$.
\end{itemize}



%%% Field Theory


\section{Field Theory}


\subsection{Equations of Motion}

\begin{itemize}
\item Dirac Equation
\begin{equation}
i\hbar \gamma^\mu \partial_\mu \psi - m c \psi = 0
\end{equation}
\item Klein Gordon Equation
\begin{equation}
\frac {1}{c^2} \frac{\partial^2}{\partial t^2} \psi - \nabla^2 \psi + \frac {m^2 c^2}{\hbar^2} \psi = 0
\end{equation}
\end{itemize}





%%% Neutrinos

\section{Neutrinos}


\subsection{Fundamental Parameters}

\begin{itemize}
\item Mixing angles
\begin{align}
\sin^2 2\theta_{12} & = 0.857 \pm 0.024 \\
\sin^2 2\theta_{23} & > 0.95 \\
\sin^2 2\theta_{13} & = 0.095 \pm 0.010 \\
\end{align}
\item Masses
\begin{align}
\Delta m_{12}^2 &= \Delta m_{sol}^2 = 7.53_{-0.18}^{+0.18} \times 10^{-5} \mathrm{eV}^2 \\
\lvert\Delta m_{31}^2\rvert & = \Delta m_{atm}^2 = 2.44_{-0.06}^{+0.06}\times 10^{-3} \mathrm{eV^2}
\end{align}

\end{itemize}

\subsection{Nuclear Reactions}


\begin{figure}[h]
\centering
\includegraphics[width=\columnwidth]{assets/Beta_Negative_Decay.eps}
\caption{Feynman diagram of beta decay. The charged current weak interaction boson in this case is a $W^-$. Credit: Joel Holdsworth, within public domain.}
\label{fig:Beta_Negative_Decay}
\end{figure}
i

\begin{table}[ht]
\centering
 \begin{tabular}{|c | c | c|} 
 \hline
 Reaction & Equation & Boson   \\ [0.5ex] 
 \hline
 Electron emission & ${}^A_Z X \to {}^A_{Z+1}X + e^- +\bar \nu_e$ & $W$  \\ 
 Positron emission & ${}^A_Z X \to {}^A_{Z-1}X + e^+ + \nu_e$ & $W$  \\
 Electron capture & ${}^A_Z X + e^- \to {}^A_{Z-1}X  + \nu_e$ &  $W$ \\
 Positron capture & ${}^A_Z X + e^+ \to {}^A_{Z+1}X  + \bar\nu_e$ &  $W$ \\
 [0.5ex] 
 \hline

 Electron annihilation &  $e^- + e^+  \to \nu_e + \bar\nu_e $  & $W$ \\
 Electron annihilation &  $e^- + e^+  \to \nu + \bar\nu $  & $Z$ \\
 [0.5ex] 
 \hline

  Neutrino capture & ${}^A_{Z}X + \overset{(-)}{\nu_e} \to {}^A_{Z\mp 1}X + e^\pm $ & W\\
  [1ex] 
 \hline
 $e^-\nu$ scattering & $e^- + \overset{(-)}{\nu_e} \to e^- + \overset{(-)}{\nu_e} $ &  $W$ \\
 $e^-\nu$ scattering & $e^{\pm} + \overset{(-)}{\nu_e} \to e^{\pm} + \overset{(-)}{\nu_e} $ &  $Z$ \\
 Neutrino scattering & $ {}^A_Z X + \overset{(-)}{\nu} \to {}^A_Z X + \overset{(-)}{\nu} $ &  Z\\
 [0.5ex] 
 \hline
 
 Bremsstrahlung & $N+N\rightleftharpoons N+N + \nu + \bar\nu$ & \\
 Annihilation & $e^+e^- \rightleftharpoons \nu + \bar \nu$   & \\
 Neutrino annihilation & $\nu + \bar \nu  \rightleftharpoons \nu + \bar \nu $   &  \\
 [0.5ex] 
 \hline
 \end{tabular}
 \caption{Neutrino related nuclear or leptonic reactions}
\label{table:Neutrino_Reactions}
\end{table}



\subsection{Neutrino Mixing}

\begin{figure}
\centering
\includegraphics[width=\columnwidth]{assets/neutrinoMixingAngle.png}
\caption{Neutrino mixing. Blue states are the VACUUM energy eigenstates while the orange states are the flavor eigenstates. Blue: electron flavor; Red: the other flavor.}
\label{fig:neutrinoMixingAngle}
\end{figure}




%%% Numerical

\section{Numerical}
Check the code step by step:

\begin{itemize}
\item Vacuum oscillation amplitude and frequency
\item Constant matter potential oscillation amplitude and frequency
\item MSW resonance
\end{itemize}




\end{multicols}

\end{document}

